\documentclass[a4paper, twoside]{article}

\setlength{\oddsidemargin}{0in} \setlength{\evensidemargin}{0in}
\setlength{\textwidth}{6.2in}
\setlength{\topmargin}{-0.3in} \setlength{\textheight}{9.8in}

\title{CS22510 Assignment - Part 2}
\author{Tom Leaman (thl5)}

\begin{document}
\maketitle
\newpage

\section{Language selection for part 1}
\subsection{Checkpoint Manager}
I decided to write the Checkpoint Manager program in Java as this allowed me to
make use of the Swing libraries to write the GUI. The Swing framework provides
an easy-to-use set of widgets which (at least in theory) is not platform
specific. This means that a single compiled version of the program can be run on
a whole host of different hardware during the event. This would mean that the
event organisers (who will probably want a computer at each monitored
checkpoint) can make use of any and all machines that can be begged and borrowed
for the event. The only requirement is that they are able to run a JVM.

\subsection{Event Manager}
I wrote the Event Manager program in C. This program is, in essence, a
refactoring of the CS127 assignment and to write this again in another language
(especially one which I am less familiar with) I felt was simply not sensible in
the time alloted.

That said, I feel that this is the most important part of the system, especially
in terms of race safety (it may be critical to know whether competitors are
still running or not) and I think it would be easier to have confidence in this
program (after a proper software audit) as every part of the system can be
examined. That is, the program makes only very minimal use of external libraries
and those that are used are well understood and trusted. If this program were to
be written in Java, for example, it would be easier to make use of a library
which may not do exactly what it claims.

\subsection{Event Creator}
My decision to write the Event Creator in C++ was essentially forced by the
specification to write each program in a different language. I have made no use
of C++'s ability to provide an Object Oriented class structure. Of the three
programs, this was the most simple and C++ is the language I am least familiar
with.

\section{My Experiences}
Having programmed in Java for nearly 2 years before arriving at University, I
feel I have a reasonably good grasp of what is and is not achievable, in
addition to a good understanding of the Swing framework. This allowed me to
write the Checkpoint Manager program reasonably quickly.

I feel strongly that Java's primary benefit is the wealth of libraries available
(for free, no less!) in addition to the inline documentation provided by
Javadoc. This makes it incredibly easy to write clean, reusable code with
accurate, helpful documentation.

My only frustration with Java is that it is quite easy (even likely) that one
will end up producing several files with little to no actual code within them.
For example, a simple interface with only a single function will add an entire
file, plus boiler-plate for the interface declaration, when in essence we only
care about the single line defining the method in question. This can lead to
a quite complicated directory structure before any useful code is actually
written. On the other hand, this does allow the programmer to make good use of
the Object Oriented-ness of Java and all that that brings with it (e.g. type
safety).

\end{document}
