\documentclass[a4paper, twoside]{article}

%\usepackage{fixltx2e}

\setlength{\oddsidemargin}{0in} \setlength{\evensidemargin}{0in}
\setlength{\textwidth}{6.2in}
\setlength{\topmargin}{-0.3in} \setlength{\textheight}{9.8in}

\title{CS22510 Assignment - Runners and Riders}
\author{Tom Leaman (thl5)}

\begin{document}
\maketitle
\newpage
\tableofcontents
\newpage

\section{Event Creator}
\subsection{Source code}
%TODO
\subsection{Compiler output}
%TODO
\subsection{Example usage}
%TODO
\subsection{Generated files}
%TODO

\section{Checkpoint Manager}
\subsection{Source code}
%TODO
\subsection{Compiler output}
%TODO
\subsection{Example usage}
%TODO

\section{Event Manager}
\subsection{Compiler output}
%TODO
\subsection{Example usage}
%TODO
\subsection{Generated results}
%TODO
\subsection{Generated log file}
%TODO

\section{Program descriptions}
\subsection{Event Creator}
The Event Creator program has been implemented in C++. It allows the user to
create a new event, add competitors to an event and create courses for an event.
It does virtually no error checking (it will crash if it is given the wrong
format for data e.g. a string instead of an int). I feel this is its greatest
short-coming.
\subsection{Checkpoint Manager}
%TODO
\subsection{Event Manager}
%TODO

\end{document}
